\documentclass[11pt]{book}
\usepackage{graphicx} % Required for inserting images
\usepackage{amsmath}
\usepackage{amsthm}
\usepackage{mathtools}
\usepackage{booktabs}
\usepackage{pdfpages}
\usepackage{yhmath}
\usepackage{enumitem}
\usepackage{multicol}
\usepackage{lmodern}
\usepackage{eufrak}
\usepackage{hyperref}
\usepackage{float}
\usepackage{verbatim}
\usepackage{cancel} 
\usepackage{mathrsfs}
\usepackage{fancyref}
\usepackage{cancel}
\usepackage{xcolor}
\usepackage{amsfonts}
\usepackage{amssymb}
\usepackage[backend=biber,style=numeric, sorting=none]{biblatex} % Modern citation style
\addbibresource{references.bib} % Specify the bibliography file
\usepackage{imakeidx}
\usepackage{xfrac}
\usepackage{titlesec}
\usepackage[toc, page]{appendix} %apéndices
\usepackage{csquotes}
\usepackage[spanish, english]{babel}
%\usepackage[a4paper, top=2cm, bottom=3cm,left=2cm,right=2cm]{geometry}

\usepackage{here}

\textheight=21cm
\textwidth=16cm
\topmargin=-1cm
\oddsidemargin=0.4cm
\pagestyle{plain}
\evensidemargin=-0.4cm
%\renewcommand{\baselinestretch}{1.2}
%\renewcommand{\labelitemii}{◦}
%\setlength{\parskip}{0em}
\setlength{\parindent}{15pt} 

\def\tq{\;\;|\;\;}
\def\qq{\quad}
\def\K{\mathbb{K}}
\def\N{\mathbb{N}}
\def\R{\mathbb{R}}
\def\Z{\mathbb{Z}}
\def\Q{\mathbb{Q}}
\def\C{\mathbb{C}}
\def\A{\mathbb{A}}
\def\proof{\square}
\def\suma{\sum\limits_{n=1}^\infty a_n}
\def\norm{\mathrm{N}}
\def\tr{\mathrm{Tr}}
\def\irr{\mathrm{Irr}}
\def\End{\mathrm{End}}
\def\car{\mathrm{car}}
\def\An{\mathrm{An}}
\def\Im{\mathrm{Im}}
\def\Ker{\mathrm{Ker}}
\def\Nil{\mathrm{Nil}}
\def\GL{\mathrm{GL}}
\def\Spec{\mathrm{Spec}}
\def\disc{\mathrm{disc}}
\def\Id{\mathrm{Id}}
\def\Mat{\mathrm{Mat}}
\def\O{\mathcal{O}}
\def\qed{\hfill\blacksquare}
\def\lcm{\mathrm{lcm}}

\newcommand\bigslant[2]{{^{\displaystyle #1}}\Big/{_{\displaystyle #2}}}
\newcommand{\Gal}[2]{{\mathrm{Gal}\left(\raisebox{.2em}{$#1$}\hspace{-0.3em}\left/\raisebox{-.2em}{$#2$}\right.\right)}}

\titlespacing{\paragraph}{%
  0pt}{%              left margin
  0em}{% space before (vertical)
  1em}%               space after (horizontal)

\DeclarePairedDelimiter\inorm{\lVert}{\rVert}%

\def\padic{\textit{p}-adic }
\def\th{^{\textrm{th}}}
\newcommand{\polmin}[2]{\textrm{polmin} ( #1,\;#2 ) }

%Numbering theorems, corollaries and lemmas.
\newtheorem{theorem}{Theorem}[section]
\newtheorem{ej}{Ejercicio}
\newtheorem*{theorem*}{Theorem}
\newtheorem{corollary}{Corollary}[theorem]
\newtheorem{lemma}{Lema}
\newtheorem{prop}[theorem]{Proposition}
\newtheorem*{reptheorem}{Theorem}

%Definitions
\theoremstyle{definition}
\newtheorem{definition}{Definition}[section]


%\setlength{\parskip}{1em} % 1ex plus 0.5ex minus 0.2ex}

\title{Práctica Anillos}
\author{David Huélamo Longás}

\begin{document}

\begin{ej}
Se considera un conjunto $R$ con dos operaciones internas $+$ y $\cdot$ tal que $(R,+)$ es un grupo, $\cdot$ es asociativa y distributiva respecto a $+$. Si $\cdot$ posee identidad, demostrar que $(R,+,\cdot)$ es un anillo.
\end{ej}
\noindent Equivale a probar que la operación $+$ es conmutativa. Por un lado tenemos\[
-(a+b)=-a-b,\quad -(b+a)=-b-a
\]Por tanto\[
-(a+b)+a=-a-b+a,\quad -(b+a)+a=-b
\]Restando:\[
-(a+b)+(b+a)=-a-b+a+b=-(a+b)+(a+b)=0
\]Por tanto,\[
a+b=b+a.
\]Hemos usado que para todo elemento $r$ de $R$, $-1\cdot c=-c$.

\begin{ej}
Sea $R$ un anillo con identidad y $a\in R$ un elemento de $R$ que no es divisor de cero a derecha y que tiene un inverso a izquierda. Demostrar que $a$ es unidad en $R$.
\end{ej}
Por hipótesis, si $r\in R$ es tal que $ra=0$, entonces $r=0$. Por otro lado, existe $s\in R$ tal que $sa=1$. Así:\[
ara=a\implies (ar-1)a=0\implies ar=1.
\]

\begin{ej}
    Demostrar que si $R$ es un anillo con identidad y $a,b\in R$ son dos elementos de $R$ tales que $1-ba$ tiene un inverso a izquierda, entonces $1-ab$ también tiene un inverso a izquierda.
\end{ej}
\noindent Sea $r\in R$ el inverso a izquierda de $1-ba$. Entonces\[
r(1-ba)=r-rba=1\implies rba=r-1
\]Por tanto\[
ab = a\cdot 1\cdot b = a[r(1-ba)]b=arb-arbab=arb(1-ab)
\]Sea $s=1+arb$. Entonces\[
s(1-ab)=(1+arb)(1-ab)=(1-ab)+arb(1-ab)=(1-ab)+ab=1
\]

\begin{ej}
    Sea $p$ un número primo. Demostrar que si $R$ es un anillo con identidad de orden $p^2$, entonces $R$ es conmutativo. ¿Es cierto si $R$ no posee identidad? Dar un ejemplo de anillo con identidad no conmutativo de orden $p^3$.
\end{ej}
\noindent Sabemos que $|Z(R)|>1$ porque $0,1\in Z(R)$. Supongamos que $|Z(R)|=p$. Entonces $R/Z(R)$ es un grupo cíclico con la suma, por tener orden $p^2/p=p$. Supongamos que este grupo viene generado por $r\in R$. Sean $x,y\in R$. Queremos ver $xy=yx$. Se tiene:\[
x+Z(R)=nr+Z(R),\quad y+Z(R)=mr +Z(R)
\]para enteros positivos $m,n$. Por tanto,\[
    x=nr+z_1,\quad y=mr +z_2
\]para elementos $z_1,z_2$ del centro de $R$. Así:\[
xy=nr\cdot mr + nr z_2 + z_1mr +z_1z_2 =nmr^2+ nr z_2 + z_1mr +z_1z_2=mnr^2+ nr z_2 + z_1mr +z_1z_2.
\]Por otro lado\[
yx=mr\cdot nr + mrz_1 + z_2nr + z_2z_1=mnr^2+ nr z_2 + z_1mr +z_1z_2=xy.
\]El caso $|Z(R)|=p^2$ implica $R=Z(R)$ luego $R$ es directamente conmutativo.

Considérese el anillo\[
\left\{
    \begin{pmatrix}
        a&b\\0&0
    \end{pmatrix}\Big|\; a,b\in \Z_p
\right\}
\]Entonces\[
e_{12}e_{11}=0\neq e_{12}=e_{11}e_{12}.
\]Por tanto es un ejemplo de anillo de orden $p^2$ sin identidad que no es conmutativo. Para un ejemplo de orden $p^3$ basta tomar el anillo\[
    \left\{
        \begin{pmatrix}
            a&b\\0&c
        \end{pmatrix}\Big|\; a,b,c\in \Z_p
    \right\}
\]Y tenemos de la misma forma que \[
    e_{12}e_{11}=0\neq e_{12}=e_{11}e_{12}.
\]
\begin{ej}
    Sea $x$ un elemento nilpotente de un anillo con identidad $R$. Prueba que $1+x$ es una unidad de $R$. Deduce que si $R$ es conmutativo, la suma de un elemento nilpotente y una unidad es una unidad. Dar un ejemplo que demuestre que 
\end{ej}
Sea $x$ tal que $x^n=0$ par algun entero positivo $n$. Sea $u=-x$. Entonces\[
(1-u)(1+u+\cdots + u^{n-1})=1-u^n=1
\]Como $1-u=1+x$, esto demuestra que $1+x$ es unidad.

Supongamos que $R$ es conmutativo. Sea $x$ nilpotente y sea $u\in R$ una unidad. Veamos que $x+u$ es una unidad. Considérese el elemento $y=-xu^{-1}\in R$. Entonces $y^n=0$ por ser $x$ nilpotente. Además,\[
(x+u)u^{-1}(1+y+\cdots + y^{n-1})=1-y^n=1.
\]Esto es lo que queríamos demostrar.

Como contraejemplo para mostrar que la conmutatividad es necesaria, considérense las matrices en $\Z_2$. En este caso, la matriz $e_{12}$ es nilpotente dado que $e_{12}^2=0$. Además, la matriz\[
\begin{pmatrix}
    1&0\\1&1
\end{pmatrix}
\]es unidad dado que su cuadrado es la identidad. Sin embargo, la suma de estas matrices tiene determinante 0, por lo que no es unidad.

\begin{ej}
    Se considera un elemento $a$ de un anillo con identidad $R$ con más de un inverso por la derecha. Demostrar que $a$ tiene infinitos inversos a derecha en $R$.
\end{ej}
\noindent Sean $b_1,b_2$ distintos tales que $ab_1=ab_2=1$. Supongamos por reducción al absurdo que hay un número finito de inversos a derecha. Sean $b_1,\dots,b_n$ todos ellos (distintos entre sí). Definamos $c_i\coloneq 1-b_ia$ para $1\leq i\leq n$. Si $c_i=c_j$, encontes\[
c_i=c_j\implies b_ia=b_ja\implies b_jab_i=b_iab_i\implies b_i=b_j
\]Es decir, que todos los $c_i$ son distintos entre sí. Además, $ac_i=0$. Por tanto $b_1+c_i$ son inversos de $a$ a izquierda:\[
a(b_1+c_i)=ab_1+ac_i=ab_1=1.
\]Por tanto\[
b_1,b_1+c_1,\dots,b_n+c_n
\]son $n+1$ inversos de $a$. Falta ver que son distintos. Ello equivale a probar que los $c_i$ son no nulos. Equivalentemente, que $1\neq b_ia$ para todo $i$. Supongamos por reducción al absurdo que $b_1a=0$. Entonces:\[
0=(1-b_1a)b_2=b_2-b_1(ab_2)=b_2-b_1\implies b_1=b_2.
\]Esto es una contradicción dado que hemos supuesto que los $b_i$ eran todos distintos.

\textbf{Prueba alternativa:} Sea $e=ba$. Entonces $e^2=baba=b(ab)a=ba=e$. Así,\[
(1-e)(1-e)=1+e^2-e-e=1+e-e-e=1-e
\]Definimos $e_{ij}\coloneq b^i(1-e)a^j$, para números naturales $i,j$. Entonces\begin{equation}\label{eq:ej-7}
e_{ij}e_{kl}=b^i(1-e)a^jb^k(1-e)a^l
\end{equation}Notamos que\[
a^jb^k=\left\lbrace\begin{array}{ll}
    a^{|j-k|}&\text{si }j\geq k\\
    b^{|j-k|}&\text{si }j<k
\end{array}\right.
\]Además, notamos que\[
a(1-e)=0=(1-e)b
\]Concluyendo así que la parte derecha de \ref{eq:ej-7} se anula salvo que $k=j$, caso en el que se tiene\[
e_{ij}e_{jl}=b^i(1-e)a^l=e_{il}
\]En resumen:\[
e_{ij}e_{kl}=\left\{\begin{array}{ll}
    0&\text{si }j\neq k\\
    e_{il} &\text{si }j=k.
\end{array}\right.
\]Notamos que son distintos porque, por ejemplo,\[
e_{ii}e_{ii}=e_{ii}\neq 0=e_{ii}e_{jj}.
\]

\begin{ej}
    Demostrar que si un anillo $R$ no tiene elementos nilpotentes distintos de cero, entonces todo elemento idempotente pertenece al centro de $R$.
\end{ej}
Sea $a\in R$ tal que $a^2=a$. Sea $x\in R$. Entonces\[
xa-axa,\; ax-axa
\]son nilpotentes:\[
(xa-axa)^2=xaxa-xaaxa-axaxa+axaaxa = 0.
\]Y análogamente\[
(ax-axa)^2=0.
\]Por hipótesis, $xa-axa=0=ax-axa$. Concluimos así que $ax=xa$, y por tanto $a\in Z(R)$. Esto ocurre, por ejemplo, en los anillos de división. Si en un anillo de división un elemento es nilpotente, entonces necesariamente es el 0.

\begin{ej}
    Sea $S$ un subanillo de un anillo $R$. Demostrar que si $R$ y $S$ tiene identidad y éstas son distintas, entonces la identidad de $S$ es un divisor de cero de $R$.
\end{ej}
Se da la siguiente situación:\[
(1_R-1_S)\cdot 1_S=1_R1_S-1_S1_S=1_S-1_S=0.
\]Sin embargo, $1_R-1_S\neq 0$ por hipótesis. Como ejemplo de esta situación, podemos considerar $R$ y $S$ como sigue:\[
S\coloneq \left\{\begin{pmatrix}
a&0\\0&0
\end{pmatrix}\Big|\; a\in\Z\right\},\quad R=\left\{\begin{pmatrix}
    a&b\\c&d
    \end{pmatrix}\Big|\; a,b,c,d\in\Z\right\}
\]

\begin{ej}
    Sea $R$ un anillo en el que para todo $x\in R$ existe $n_x>1$ tal que $x^{n_x}=x$. Demostrar que $R$ no tiene elementos nilpotentes no nulos y que si $R$ es conmutativo con identidad, todo ideal primo es maximal.
\end{ej}
Sea $0\neq x\in \Nil(R)$. Entonces existe un entero positivo $m$ tal que $x^m=0$. Supongamos que $m$ es el menor entero positivo satisfaciendo tal propiedad. Como $x$ es no nulo, necesariamente $m\neq n_x$. Distinguimos casos: Si $m<n_x$, entonces \[
x^{n_x}=x^mx^{n_x-m}=0,
\]lo que supone una contradicción. Por otro lado, si $m\geq n_x$, podemos aplicar el algoritmo de la división\[
m=n_xq+r
\]para enteros $q,r$ y $0\leq r<n_x$. Así:\[
0=x^m=x^{n_xq+r}=x^{q+r}
\]Dado que $n_x>1$, necesariamente $q+r<m$. Esto contradice la propiedad de minimalidad de $m$. Esto nos permite concluir que el anillo no tiene elementos nilpotentes no nulos.

Para el siguiente paso del problema necesitamos usar el siguiente lema:
\begin{lemma}\label{lema-1}
    Si en el enunciado del problema suponemos que $R$ es un dominio de integridad, entonces $R$ es directamente un cuerpo.
\end{lemma}
\noindent\textit{Demostración.} Supongamos que $R$ es dominio de integridad. Sea $x$ un elemento no nulo de $R$. Entonces $x^{n_x}-x=x(x^{n_x-1}-1)=0$ y dado que $x$ es no nulo, necesariamente $x^{n_x-1}=1$. Por tanto $x$ es unidad. Esto implica que $R$ es cuerpo.$\qed$

Sea ahora $P$ un ideal primo y que $R$ es conmutativo. Entonces $R/P$ es un dominio de integridad. Así, para todo elemento $x$ de $R$,\[
(x+P)^{n_x}=x+P,
\]por el lema anterior, esto implica que $R/P$ es cuerpo. Así, $P$ es ideal maximal.

Para próximos ejercicios será interesante tener en cuenta el siguiente lema:\begin{lemma}\label{lema-2}
    Bajo las condiciones del ejercicio anterior, para todo $x$ elemento de $R$, $x^{n_x-1}$ pertenece al centro del anillo $R$
\end{lemma}
\noindent\textit{Demostración.}Se tiene la siguiente situación:\[
(x^{n_x-1})^2=x^{n_x-1}\cdot x^{n_x-1}=x^{n_x}x^{n_x-2}=x^{n_x-1},
\]por lo que $x^{n_x-1}$ es nilpotente. Por el ejercicio 7, $x^{n_x-1}$ pertenece al centro de $R$.$\qed$

\begin{ej}
    Sea $R=\{f\in \R^{[0,1]}:\; f\text{ es continua}\}$. Demuestra que $R$ es un anillo conmutativo con identidad y determina sus ideales maximales.
\end{ej}
\noindent El anillo es conmutativo porque lo es $R$. Además, su identidad es la aplicación constante 1. Vamos a determinar sus ideales maximales. Para cada $\alpha\in [0,1]$, considérese la aplicación\[
\begin{array}{rrcl}
    \phi_\alpha:&R&\longrightarrow&\R\\
    &f&\longmapsto& f(\alpha).
\end{array}
\]Esta aplicación es un homomorfismo de anillos cuyo núcleo es \[
M(\alpha):=\ker\phi_\alpha=\{f\in R:\; f(\alpha)=0\}\neq R.
\]Sea $M$ un ideal maximal tal que para todo $\alpha\in [0,1]$, $M\nsubseteq M(\alpha)$. Entonces para cada $\alpha\in[0,1]$ existe una función $f_\alpha\in M$ tal que $f_\alpha(\alpha)\neq 0$. Como $f$ es continua, considérese $U_\alpha$ un entorno de $\alpha$ contenido en $[0,1]$ en el que $f_\alpha$ no se anula. Como $[0,1]$ es compacto, podemos recubrirlo con un número finito de estos entornos $\{U_{\alpha_1},\dots, U_{\alpha_n}\}$. Definimos\[
h\coloneq f_{\alpha_1}^2+\cdots + f_{\alpha_n}^2\in M.
\]Entonces $h$ no se anula en ningún punto del intervalo $[0,1]$. Por tanto $h$ es una unidad que está en $M$. Concluimos así que $M=R$. Pero esto contradice la condición de maximalidad de $M$. Por tanto, los ideales maximales son los de la forma $M(\alpha)$ para algún $\alpha\in [0,1]$.

\begin{ej}
    Si $x^3=x$ para todo elemento $x$ de $R$, demostrar que $6x=0$ para todo $x\in R$ y que $R$ es conmutativo.
\end{ej}
Para todo $x$ de $R$, $2x=(2x)^3=8x^3=8x$ y por tanto $6x=0$. Además, por el lema \ref{lema-2}, $x^2$ pertenece al centro de $R$ para todo $x$ de $R$. Así, dados $x,y$ elementos de $R$,\[
xy=(xy)^3=xyxyxy=x(yx)^2y=xy(yx)^2=xy^2xyx=y^2x^2yx=y^3x^3=yx,
\]concluyendo así que $R$ es conmutativo.

\begin{ej}
    Sea $R$ anillo con identidad, y sea $S=\mathrm{Mat}_n(R)$. Demostrar que los ideales de $S$ son exactamente los de la forma $\mathrm{Mat}_m(I)$ para $I$ un ideal de $R$. Determinar los ideales primos y maximales de $S$.
\end{ej}
Sea $J$ un ideal de $S$. Consideramos el siguiente subconjunto de $R$:\[
I\coloneq\{r\in R|\;\exists (a_{ij})\in J:\;a_{11}=r\}.
\]Veamos que $I$ es ideal de $R$. La suma de dos elementos de $I$ está claramente en $I$. Además, al multiplicar un elemento de $a\in I$ por cualquier elemento de $r\in R$, consideramos $(a_{ij})\in J$ tal que $a_{11}=a$, y entonces la matriz $(b_{ij})$ dada por $b_{ij}=ra_{ij}$ satisface $b_{11}=ra$. Así, $I$ es un ideal de $R$. Veamos que $J=\mathrm{Mat}_n(I)$ por doble inclusión.

Sea $(b_{ij})\in \mathrm{Mat}_n(I)$. Como $b_{ij}\in I$, por definición existe para cada $i,j\in\{1,\dots,n\}$ una matriz $A_{ij}$ de $J$ cuya primera entrada es igual a $b_{ij}$. Así, \[
B_{ij}\coloneq e_{i1}A_{ij}e_{1j}
\]es una matriz de $J$ (por ser $J$ ideal) con entradas todas nulas salvo la entrada $i,j$, que será igual a $b_{ij}$. Por tanto,\[
(b_{ij})=\sum_{i,j\in\{1,\dots, n\}}B_{ij}\in J.
\]Concluimos que $\mathrm{Mat}_n(I)\subseteq J$.

Sea ahora $(a_{ij})\in J$. Queremos ver que todas las entradas de $(a_{ij})$ están en $I$. Para $i_0,j_0\in\{1,\dots,n\}$ dados, la matriz\[
e_{1i_0}(a_{ij})e_{j_01}
\]tiene como primera entrada $a_{i_0j_0}$, por lo que $a_{i_0j_0}\in I$. Concluimos así con la igualdad deseada.

Ahora veamos que si $I$ es un ideal de $R$, entonces $J=\mathrm{Mat}_n(I)$ es un ideal de $S$. Está claro que la suma de elementos de $J$ está en $J$. Asimismo, al multiplicar un elemento de $J$ por cualquier elemento de $S$, obtendremos una nueva matriz cuyas entradas son sumas de elementos de la forma $r\cdot a$ con $r\in R$ y $a\in I$. Como $I$ es ideal, $ra\in I$, luego las entradas de dicha matriz estarán en $I$. Por tanto $J$ es un ideal.

De esta forma, podemos considerar una biyección $\overline{\phantom{I}}$ de los ideales de $R$ a los ideales de $S$ dado por $\overline I=\Mat(I)$

Veamos ahora cuáles son los ideales primos de $S$. Para ello, necesitamos usar lo que demostraremos a continuación. Supongamos que $I\subseteq J\subseteq R$ son ideales de $R$. Entonces obviamente $\overline I\subseteq\overline J\subseteq S$. Recíprocamente, si $I,J\subseteq R$ son ideales de $R$ tales que $\overline I\subseteq \overline J$, entonces dado $a\in I$, podemos considerar la matriz $ae_{11}\in\overline I\subseteq\overline J$, concluyendo así que $a\in J$. En conclusión, $I\subseteq J$. Podemos resumir este resultado con las palabras ``$\overline{\phantom{I}}$ y su inversa conservan la inclusión de ideales''.

Vamos a ver que los ideales primos de $S$ son exactamente $\overline P$ donde $P$ es un ideal primo de $R$. Sea $P$ un ideal primo de $R$, y supongamos que $\overline I\overline J\subseteq\overline P$. Sean $a\in I$, $b\in J$. Entonces $ae_{11}be_{11}=abe_{11}\in \overline I\overline J\subseteq\overline P$, por lo que $ab\in P$. Esto implica que $IJ\subseteq P$. Así, $I\subseteq P$ o $J\subseteq P$ por ser $P$ primo. Como $\overline{\phantom{I}}$ conserva la inclusión de ideales, $\overline I\subseteq \overline P$ o $\overline J\subseteq\overline P$. Concluimos que $P$ es ideal primo de $S$.

Recíprocamente, si $\overline P$ es ideal primo de $S$ y $I,J$ son ideales de $R$ tales que $IJ\subseteq P$, entonces $\overline{IJ}\subseteq\overline P$. Notamos que $\overline I\overline J\subseteq \overline IJ$, es decir, el producto de una matriz con entradas en $I$ por una matriz con entradas en $J$ será una matriz con entradas en $IJ$. Por tanto, $\overline I\overline J\subseteq \overline P$. Como $\overline P$ es primo, necesariamente $\overline I\subseteq\overline P$ o $\overline J\subseteq\overline P$. Por tanto, $I\subseteq P$ o $J\subseteq P$. Concluimos así que $P$ es ideal primo de $R$.

Para los maximales hacemos algo similar. Los maximales de $S$ serán exactamente de la forma $\overline M$ para $M$ ideal maximal de $R$. En efecto, si $M$ es un maximal de $R$, entonces $\overline M\neq S$ y si $\overline M\subsetneq \overline I\subseteq S$, entonces $M\subsetneq I\subseteq R$ y por la maximalidad de $M$, $I=R$, luego $\overline I=S$, concluyendo así que $\overline M$ es maximal.

Recíprocamente, si $\overline M$ es maximal en $S$, entonces $\overline M\neq S$ luego $M\neq R$; y si $M\subsetneq I\subseteq R$ entonces $\overline M\subsetneq \overline I\subseteq S$, y por la maximalidad de $\overline M$ se tiene $\overline I=S$. Por tanto, $I=R$, concluyendo que $M$ es maximal de $R$.

\begin{ej}
    Sea $R=R_1\times\cdots\times R_n$ un producto de anillos con identidad. Demuestra que para todo ideal $I$ de $R$ existen ideales $I_1,\dots,I_n$ de $R_1,\dots,R_n$, respectivamente, tales que $I=I_1\times\cdots\times I_n$. ¿Es cierto el resultado si los anillos $R_i$ no tienen identidad?
\end{ej}
Sea $J$ un ideal de $R$. Consideramos el epimorfismo de anillos ``proyección'' $\pi_k:R\longrightarrow R_k$, y el ideal $I_k=\pi_k(J)$ para $1\leq k\leq n$, que es un ideal de $R$. Veamos $J=I_1\times\cdots\times I_n$. Sea $j\in J$. Entonces $\pi_k(j)\in I_k$. Así, $j\in I_1\times\cdots\times I_n$. Sea ahora $(i_1,\dots,i_n)\in I_1\times\cdots\times I_n$. Entonces $i_k=\pi_k(j_k)$ para algún $j_k\in J$. Sea $e_k\in R$ el elemento cuyas entradas son todas nulas salvo la entrada $k$-ésima, cuyo valor es 1. Este elemento existe porque todos los $R_i$ tienen identidad. Entonces como $J$ es ideal, $e_kj_k\in J$. Además,\[
(i_1,\dots,i_n)=\sum_{k=1}^n e_kj_k\in J.
\]Concluimos así con la igualdad de conjuntos.

Recíprocamente si $I_k$ es un ideal de $R_k$ para $1\leq k\leq n$, y consideramos el subconjunto $J=I_1\times\cdots\times I_n$ de $R$, es fácil ver que la suma de elementos de $J$ está en $J$ y que si multiplicamos un elemento de $J$ por uno de $R$ volvemos a caer en $J$. Es decir, que $J$ es un ideal de $R$.

Si consideramos el anillo cero $(\Z_2,+,\cdot)$, y definimos $R=\Z_2\times \Z_2$, entonces tomando $0\neq a\in\Z_2$, el ideal $\langle(a,a)\rangle$ de $R$ no es producto de ideales de $\Z_2$, dado que cualquier producto de ideales de $\Z_2$ es el ideal 0, mientras que el primero no el ideal cero. Por tanto, si los anillos no tienen identidad, el resultado no es cierto.

\begin{ej}
    Sea $f:\R\longrightarrow S$ un homomorfismo de anillos. Sea $I$ un ideal de $R$ y $J$ un ideal de $S$. Demuestra que\begin{enumerate}
        \item Si $f$ es un epimorfiso y $J$ es maximal en $S$, entonces $f^{-1}(J)$ es ideal maximal de $R$.
        \item Si $f$ es epimorfismo y $\ker f\subseteq I$ e $I$ es ideal maximal de $R$, entonces $f(I)$ es un ideal maximal de $S$.
        \item Si $J$ es ideal primo de $S$ y $f$ es epimorfismo, entonces $f^{-1}(J)$ es ideal primo de $R$.
        \item Si $f$ es epimorfismo, $\ker f\subseteq I$ e $I$ es ideal primo de $R$, entonces $f(I)$ es ideal primo de $S$.
    \end{enumerate}
\end{ej}
\noindent Recordamos algunas propiedades básicas de las aplicaciones y de las aplicaciones sobreyectivas. Si $f:A\longrightarrow B$ es una apicación y $C\subseteq A$ y $D\subseteq B$, entonces\[
    C\subseteq f^{-1}(f(C)),\quad f(f^{-1}(D))\subseteq D
\]Además, si $f$ es sobreyectiva,\[
f(f^{-1}((D))=D,
\]y aunque no lo usaremos, si $f$ es inyectiva\[
C=f^{-1}(f(C))
\]
\begin{enumerate}
    \item Notamos que $f^{-1}(J)$ es ideal de $R$. Además, si $f^{-1}(J)=R$ entonces $f(f^{-1}(J))=J=f(R)=S$, lo que contradice la maximalidad de $J$. Así $f^{-1}(J)\subsetneq R$. Supongamos $f^{-1}(J)\subsetneq I\subseteq R$. Tomando imágenes,\[
    J\subseteq f(I)\subseteq S
    \]Supongaos que $J=f(I)$. Entonces $f^{-1}(J)=f^{-1}(f(I))\supseteq I$, lo cual es una contradicción con que $f^{-1}(J)\subsetneq I$. Así,\[
        J\subsetneq f(I)\subseteq S,
    \]y por la maximalidad de $J$, encontramos que $f(I)=S$. Así, dado $r\in R$, $f(r)\in S$, luego $f(r)=f(i)$ para algún $i\in I$, y así $f(r)-f(i)=f(r-i)=0$, es decir $r-i\in \ker f$. Como $\ker f\subseteq f^{-1}(J)\subsetneq I$, se tiene que $r-i\in I$, luego $r\in I$. Concluimos que $I=R$.

    \item Notamos que al ser $f$ sobreyectiva, $f(I)$ es ideal de $S$. Supongamos que $f(I)=S$. Entonces, razonando como anteriormente, $I=R$, lo que contradice la maximalidad de $I$. Así, $f(I)\neq S$. Supongaos ahora que $f(I)\subsetneq J\subseteq S$. Entonces\[
    I\subseteq f^{-1}(f(I))\subseteq f^{-1}(J)\subseteq S.
    \]Supongamos $I=f^{-1}(J)$. Entonces $J=f(f^{-1}(J))=f(I)$, lo que contradice el hecho de que $f(I)\subsetneq J$. Por tanto, teniendo en cuenta además que $f^{-1}(J)$ es un ideal de $R$ e $I$ es maximal, necesariamente $f^{-1}(J)=S$.

    \item Notamos que $f^{-1}(J)$ es ideal de $R$. Supongamos $f^{-1}(J)=R$. Entonces $J=f(f^{-1}(J))=f(R)=S$, lo cual es una contradicción. Sean $A,B$ ideales de $R$ tales que $AB\subseteq f^{-1}(J)$. Entonces $f(AB)\subseteq f(f^{-1}(J))=J$. Además, como $f$ es homomorfismo,\[
    f(A)f(B)\subseteq f(AB)
    \]Luego $f(A)f(B)\subseteq J$. Tanto $f(A)$ como $f(B)$ son ideales por ser $f$ un epimorfismo. Por la primalidad de $J$, $f(A)\subseteq J$ o $f(B)\subseteq J$. Concluimos así que $A\subseteq f^{-1}(f(A))\subseteq f^{-1}(J)$ o $B\subseteq f^{-1}(f(B))\subseteq f^{-1}(J)$. Por tanto $f^{-1}(J)$ es ideal primo.

    \item Notamos que $f(I)$ es ideal por ser $f$ epimorfismo. Además, si $f(I)=S$ entonces, como ya hemos argumentado anteriormente, $I=R$, lo que contradice la primalidad de $I$. Por tanto $f(I)\neq S$. Supongamos que $A,B$ son dos ideales de $S$ tales que $AB\subseteq f(I)$. Entonces $f^{-1}(AB)\subseteq f^{-1}(f(I))$. Veamos que $f^{-1}(f(I))=I$. Está claro que $I\subseteq f^{-1}(f(I))$. Sea $a\in f^{-1}(f(I))$. Entonces $f(a)\in f(I)$. Sea $i\in I$ tal que $f(a)=f(i)$. Entonces $a-i\in\ker f\subseteq I$, por lo que $a\in I$. Así, se da la igualdad de conjuntos. Por otro lado, si $rs\in f^{-1}(A)f^{-1}(B)$, entonces \[
    f(rs)\in f(f^{-1}(A)f^{-1}(B))=f(f^{-1}(A))f(f^{-1}(B))=AB,
    \]es decir, $rs\in f^{-1}(AB)$. Esto demuestra que $f^{-1}(A)f^{-1}(B)\subseteq f^{-1}(AB)$, y por tanto\[
        f^{-1}(A)f^{-1}(B)\subseteq f^{-1}(AB)\subseteq f^{-1}(f(I))=I.
    \]Por la primalidad de $I$, o bien $f^{-1}(A)\subseteq I$ o $f^{-1}(B)\subseteq I$. Tomando imágenes en ambos lados, obtenemos que $A\subseteq f(I)$ o $B\subseteq f(I)$. Esto demuestra la primalidad de $f(I)$.
\end{enumerate}

\begin{ej}
    Se considera un anillo con identidad $R$ que satisface la siguiente proipedad: para todo $r\in R$ existe un único $s\in R$ tal que $rsr=r$. Demostrar que en este caso $srs=s$ y que $R$ es un anillo de división.
\end{ej}
\noindent Tenemos la siguiente situación \[
r(srs)r=(rsr)sr=rsr=r.\]
Como $s$ es el único elemento que satisface $rsr=r$, y además $r(srs)r=r$, necesariamente $s=srs$. Por otro lado,\[
(sr)(sr)(sr)=s(rsr)sr=srsr=s(rsr)=sr, \quad (sr)\cdot 1\cdot (sr)=s(rsr)=sr.
\]De nuevo, por la unicidad, deducimos que $sr=1$.
\end{document}