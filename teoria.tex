\documentclass[11pt]{book}
\usepackage{graphicx} % Required for inserting images
\usepackage{amsmath}
\usepackage{amsthm}
\usepackage{mathtools}
\usepackage{booktabs}
\usepackage{pdfpages}
\usepackage{yhmath}
\usepackage{enumitem}
\usepackage{multicol}
\usepackage{lmodern}
\usepackage{eufrak}
\usepackage{hyperref}
\usepackage{float}
\usepackage{verbatim}
\usepackage{cancel} 
\usepackage{mathrsfs}
\usepackage{fancyref}
\usepackage{cancel}
\usepackage{xcolor}
\usepackage{amsfonts}
\usepackage{amssymb}
\usepackage[backend=biber,style=numeric, sorting=none]{biblatex} % Modern citation style
\addbibresource{references.bib} % Specify the bibliography file
\usepackage{imakeidx}
\usepackage{xfrac}
\usepackage{titlesec}
\usepackage[toc, page]{appendix} %apéndices
\usepackage{csquotes}
\usepackage[spanish, english]{babel}
%\usepackage[a4paper, top=2cm, bottom=3cm,left=2cm,right=2cm]{geometry}

\usepackage{here}

\textheight=21cm
\textwidth=16cm
\topmargin=-1cm
\oddsidemargin=0.4cm
\pagestyle{plain}
\evensidemargin=-0.4cm
%\renewcommand{\baselinestretch}{1.2}
%\renewcommand{\labelitemii}{◦}
%\setlength{\parskip}{0em}
\setlength{\parindent}{15pt} 

\def\tq{\;\;|\;\;}
\def\qq{\quad}
\def\K{\mathbb{K}}
\def\N{\mathbb{N}}
\def\R{\mathbb{R}}
\def\Z{\mathbb{Z}}
\def\Q{\mathbb{Q}}
\def\C{\mathbb{C}}
\def\A{\mathbb{A}}
\def\proof{\square}
\def\suma{\sum\limits_{n=1}^\infty a_n}
\def\norm{\mathrm{N}}
\def\tr{\mathrm{Tr}}
\def\irr{\mathrm{Irr}}
\def\End{\mathrm{End}}
\def\car{\mathrm{car}}
\def\An{\mathrm{An}}
\def\Im{\mathrm{Im}}
\def\Ker{\mathrm{Ker}}
\def\Nil{\mathrm{Nil}}
\def\GL{\mathrm{GL}}
\def\Spec{\mathrm{Spec}}
\def\disc{\mathrm{disc}}
\def\Id{\mathrm{Id}}
\def\O{\mathcal{O}}
\def\Rad{\mathrm{Rad}}
\def\Max{\mathrm{Max}}
\def\J{\mathrm{J}}
\def\qed{\hfill\blacksquare}
\def\lcm{\mathrm{lcm}}

\newcommand\bigslant[2]{{^{\displaystyle #1}}\Big/{_{\displaystyle #2}}}
\newcommand{\Gal}[2]{{\mathrm{Gal}\left(\raisebox{.2em}{$#1$}\hspace{-0.3em}\left/\raisebox{-.2em}{$#2$}\right.\right)}}

\titlespacing{\paragraph}{%
  0pt}{%              left margin
  0em}{% space before (vertical)
  1em}%               space after (horizontal)

\DeclarePairedDelimiter\inorm{\lVert}{\rVert}%

\def\padic{\textit{p}-adic }
\def\th{^{\textrm{th}}}
\newcommand{\polmin}[2]{\textrm{polmin} ( #1,\;#2 ) }

%Numbering theorems, corollaries and lemmas.
\newtheorem{theorem}{Teorema}[section]
\newtheorem*{theorem*}{Teorema}
\newtheorem{corollary}{Corolario}[theorem]
\newtheorem{lemma}[theorem]{Lema}
\newtheorem{prop}[theorem]{Proposición}
\newtheorem*{reptheorem}{Teorema}

%Definitions
\theoremstyle{definition}
\newtheorem{definition}{Definición}[section]


%\setlength{\parskip}{1em} % 1ex plus 0.5ex minus 0.2ex}

\title{TFG}
\author{David Huélamo Longás}

\begin{document}
\chapter{Anillos}
\begin{enumerate}
    \item Sean $R_1,\dots,R_n$ anillos. El producto cartesiano\[
    S=R_1\times\cdots\times S_N
    \]tiene estructura de anillo con las operaciones evidentes. Además,\[
    \overline{R_i}:=\{0\}\times\cdots\times R_i\times\cdots\times \{0\}\subseteq S
    \]es un subanillo.

    El anillo $S$ es conmutativo si y sólo si para todo $i=1,\dots, n$, $R_i$ es conmutativo. Si para todo $i$, $R_i$ tiene identidad, entonces $S$ tiene identidad y ésta es \[
    1_S=(1_{R_1},\dots,1_{R_n}).
    \]Notamos que la identidad de $\overline{R_i}$ es\[
    (0,\dots,0,1_{R_i},0,\dots,0),
    \]y si $n$ es mayor o igual que 2, no coincide con la identidad de $S$ para ningún $i$. Notamos también que si $i\neq j$, entonces\[
    1_{\overline{R_i}}\cdot 1_{\overline{R_j}}=0,
    \]es decir, $S$ tiene divisores de 0. En particular, el producto cartesiano de cuerpos no es cuerpo.

    \item Sea $R$ un anillo, y $X\neq \emptyset$ un conjunto. Definimos\[
    R^X:=\{f:X\longrightarrow R:\; f\text{ aplicación}\}.
    \]$R^X$ es un anillo con las operaciones de suma y producto de funciones:\[
    (f+g)(x)=f(x)+g(x),\quad (f\cdot g)(x)=f(x)\cdot g(x).
    \]Si $R$ tiene identidad, la aplicación $1(x)=1_R$ es la identidad en $R^X$. Además, $R$ es conmutativo si y sólo si $R^X$ también lo es.

    Si $|X|\geq 2$ y $|R|\geq 2$, $R^X$ tiene divisores de 0: Sea $a$ un elemento no nulo de $R$ y $x_1,x_2$ dos elementos distintos de $X$. Entonces las funciones $f_a$, $\overline{f_a}$ dadas por\[
    \begin{array}{ll}
        f_a(x_1)=a, & f_a(z)=0\text{ si }z\neq x_1,\\
        \overline{f_a}(x_2)=a, & \overline{f_a}(z)=0\text{ si }z\neq x_2,
    \end{array}
    \]son funciones no nulas cuyo producto es la función cero.

    Nótese que si $X=\N$, el conjunto $R^X$ es el anillo de sucesiones en $R$.

    \item Sea $R$ un anillo con identidad y sea $(G,\cdot)$ un grupo (¿finito?). Definimos el \textit{anillo de grupo}\[
    RG=\left\{\sum_{g\in G}a_g\cdot g \;\Big|\; a_g\in R,\; g\in G\right\}.
    \]con las operaciones de suma y producto dadas por\[\begin{split}
        &\sum_{g\in G}a_g g+\sum_{g\in G}b_g g=\sum_{g\in G}(a_g+b_g)g,\\
        &\left(\sum_{g\in G}a_g g\right)\cdot \left(\sum_{x\in G}b_x x\right)=\sum_{x\in G}\left(\sum_{g\in G}(a_gb_{g^{-1}x})\right)x.
    \end{split}
    \]Vamos a ver por qué se define así el producto. Si trabajamos con elementos de la forma $a\cdot g$ y $b\cdot h$, es natural que el producto se defina como\[
    (a g)\cdot(b h) = (ab)\cdot (gh).
    \]Generalizando,\[
    (a g)\cdot \left(\sum_{x\in G}b_x x\right)=\sum_{x\in G}(ab_x)(gx)=\sum_{z\in G}(ab_{g^{-1}z})z=\sum_{z\in G}(ab_{g^{-1}x})x
    \]Por último,\[
        \left(\sum_{g\in G}a_g g\right)\cdot \left(\sum_{x\in G}b_x x\right)=\sum_{g\in G}\sum_{x\in G}(a_gb_{g^{-1}x})x=\sum_{x\in G}\left(\sum_{g\in G}(a_gb_{g^{-1}x})\right)x
    \]Nótese que los elementos cero e identidad del anillo son\[
    0=\sum_{g\in G}a_G\cdot g,\quad 1=1_R1_G+\sum_{g\in G-\{1_G\}}0\cdot g.
    \]y el elemento opuesto de\[
    a = \sum_{g\in G}a_g g
    \]viene dado por\[
    -a = \sum_{g\in G}-a_gg.
    \]Si consideramos el elemento\[
    a = \sum_{g\in G}g,
    \]se tiene que para todo $x$ en $G$,\[
    xa=\sum_{g\in G}xg = \sum_{g\in G}g=a.
    \]En particular, \[
    a^2=\left(\sum_{x\in G}x\right)a=\sum_{x\in G}xa = \sum_{x\in G} a = |G| a.
    \]Ahora distinguimos casos según la característica de $R$. Si $\car R$ divide a $|G|$, se tiene que $a^2=0$, por lo que $R$ tiene divisores de 0. Si $\car R\neq 0$ y $|G|$ son coprimos entre sí, entonces por la identidad de Bezout existen enteros $m,n$ tales que $m|G|+n\car G=1$. Por tanto, si tomamos\[
    b:=\overbrace{1_S+\dots+1_S}^{m\text{ veces}},\quad c=\overbrace{1_S+\dots+1_S}^{n\text{ veces}},
    \]se tiene $b|G|=1-c\car G=1$. Así, definido\[
    \lambda = b\cdot a,
    \]se tiene\[
    \lambda^2=b\cdot a\cdot b\cdot a=b^2a^2=b^2a^2=b^2(|G|a)=ba=\lambda.
    \]Donde hemos usado que $b$ conmuta con $a$, y realmente con cualquier otro elemento de $RG$ (esto es fácil de comprobar). A los elementos cuyo cuadrado coincide con ellos mismos se los denomina \textit{idempotentes}. Si a estas condiciones le añadimos $G\neq 1$, obtenemos\[
    a^2-a=a(a-1)=0,
    \]siendo tanto $a$ como $a-1$ elementos no nulos.

    Por último, notamos que si $|G|=m$, para cualquier elemento $g$ de $G$, $g^m=1$. Por tanto, si $g\neq 1$,\[
        (1-g)(1+g+\cdots+g^{m-1})=1-g^m=0,
    \]siendo ambos factores distintos de cero. Es otra forma de ver que $RG$ tiene divisores de 0, incluso si $\car R=0$.

    \item Sea $(A,+)$ un grupo abeliano y $R=\End(A)$. Entonces $R$ es un anillo con las operacioes\[
    (f+g)(x)=f(x)+g(x),\quad (f\cdot g)(x)=f(g(x)).
    \]En particular, consideramos $S=R^\N$, $T=\End(S)$. Sea $f\in T$ dado por\[
    f((a_i)_{i\geq 1})=(b_j)_{j\geq 1},\quad b_1=0,\; b_{j+1}=a_j, j\geq 2,
    \]de modo que $f$ es inyectiva pero no sobreyectiva. Consideramos la función $g$ dada por\[
    g((a_i)_{i\geq 1})=(b_j)_{j\geq 1},\quad b_j=a_{j+1},j\geq 1
    \]de modo que $f$ es sobreyectiva pero no inyectiva. Entonces\[
    g\circ f= id_T,\quad f\circ g\neq id_T
    \]Es decir, hay elementos con inversos a izquierda que no son inversos a derecha.

    \paragraph{Nota:} Sea $R$ anillo con identidad, y sea $a\in R$ tal que existen $x,y\in R$ satisfaciendo\[
        xa = 1 = ay.
    \]Entonces $x=y$:\[
        x=x\cdot 1 = x(ay)= (ax)y = 1\cdot y = y.
    \]

    \item $X:=\{1,i,j,k\}$, y sea $D$ el $\R$-espacio vectorial con base $X$. Es decir,\[
    D=\R\oplus\R i\oplus\R j\oplus\R k.
    \] Definimos la operación:\[\begin{split}
        1\cdot x = x\cdot 1 = x,\quad x\in\{i,j,k\},\quad i^2=j^2=k^2=-1,\\
        ij = k,\; jk=i,\; ki=j,\; ji=-k,\; kj=-i,\; ik=-j.
    \end{split}
    \]Esta operación se extiende a un producto en $D$\[
    (ax)(by) = (ab)(xy),\quad a,b\in\R,\; x,y\in X.
    \]y se extiende por distributividad a todo $D$. Sea $z\in D$,\[
    z=a+bi+cj+dk\neq 0,\quad a,b,c,d\in\R.
    \]entonces alguno de los números $a,b,c,d$ es no nulo. Por tanto\[
        a^2+b^2+c^2+d^2\neq 0.
    \]Así,\[
    z\cdot (a-bi-cj-dk)=a^2+b^2+c^2+d^2=:l.
    \]Definimos\[
    y\coloneq \frac{a}{l}-\frac{b}{l}i-\frac{c}{l}j - \frac{d}{l}k.
    \]Entonces\[
    z\cdot y=y\cdot z=1.
    \]Por lo que $D$ es un anillo de división que no es cuerpo.

    Si $R=\R$ y $G=Q_8$, entonces notamos que $RG=\R Q_8$ tiene ``dimensión 8'' por ser $RG$ el conjunto de las sumas formales sobre un conjunto de 8 elementos, $Q_8$, mientras que\[
    \R\oplus\R i\oplus \R j\oplus \R k
    \]tiene dimensión 4. Es decir, son anillos distintos.
\end{enumerate}
\begin{definition}
    Sea $R$ un anillo, $a\in R$. Sea \[
    \begin{array}{rlcl}
        \phi_a: & R & \longrightarrow & R\\
        & x & \longmapsto & ax
    \end{array}
    \]Entonces $\phi_a$ es un endomorfismo de $R$ como grupo abeliano. Podemos definir\[
    \begin{array}{rlcl} 
        \phi:&R&\longrightarrow &\End(R)\\
        &a&\longmapsto &\phi_a  
    \end{array}
    \]Se puede comprobar que $\phi$ es homomorfismo de anillos. El núcleo de este homomorfismo se denomina \textit{anulador de $a$}.\[
    \textrm{an}_R(a)=\ker\phi=\{a\in R:\; a\cdot x=0\text{ para todo }x\in R\}.
    \]
\end{definition}

\begin{definition}
    Sea $X\subseteq R$, donde $R$ es un anillo. Entonces el \textit{anulador de $X$} es\[
        \textrm{an}_R(X)=\{a\in R:\; a\cdot x=0\text{ para todo }x\in X\}.
    \]
\end{definition}

\begin{definition}
    Sea $R$ un anillo con identidad. Un elemento $a\in R$ es \textit{unidad} si existe $b\in R$ tal que\[
        ab=ba=1.
        \]Denotamos por $U(R)$ al conjunto de unidades de $R$.
\end{definition}

\begin{prop}
    Si $R$ es un anillo con identidad, $(U(R),\cdot)$ es un grupo.
\end{prop}
\noindent\textit{Demostración.} Sea $R$ un anillo con identidad. El elemento neutro del grupo será el $1\in U(R)$. Sean $a,b\in U(R)$. Entonces $a^{-1},b^{-1}\in U(R)$. Así,\[(ab)^{-1}=b^{-1}a^{-1}\in U(R),\]por lo que $U(R)$ es cerrado para el producto. Si $a\in U(R)$, su inverso para el producto es evidentemente $a^{-1}\in U(R)$. Notamos que $(a^{-1})^{-1}=a$. La propiedad asociativa se hereda de $R$. Por tanto, $(U(R),\cdot)$ es un grupo. $\qed$

Veamos algunos ejemplos:\begin{enumerate}
    \item Las unidades de $\Z$ son $U(\Z)=\{1,-1\}$.
    \item Si $R$ es un anillo con identidad,\[
    U(\mathrm{M}_n(R))=\GL(n,R).
    \]
    \item Si $D=\R\oplus \R i\oplus \R j\oplus \R k$, y tomamos\[
    A=\Z \oplus\Z i\oplus\Z j\oplus\Z k\subseteq D,
    \]entonces $(A,+,\cdot)$ es un anillo y $U(A)=\{\pm1,\pm i,\pm j,\pm k\}$.
    \item Si $R$ es un anillo con identidad y $G$ es finito,\[
    U(R)\cup G\subseteq U(RG).
    \]
    \item El anillo $\Z[i]$ de enteros de Gauss, dado por\[
    \Z[i]=\{a+bi:\; a,b\in\Z\}
    \]con las operaciones heredadas del cuerpo $\C$, tiene unidades $U(\Z[i])=\{\pm 1,\pm i\}\cong C_4$.
\end{enumerate}

\begin{definition}
    Sea $R$ un anillo, y sea $0\neq a\in R$. Se dice que $a$ es \textit{divisor de 0} a izquierda si $\exists b\neq 0$ tal que $ab=0$. Análogamente, se dice que $a$ es \textit{divisor de 0} a derecha si $\exists b\neq 0$ tal que $ba=0$. Si $a$ es divisor de 0 a izquierda y a derecha, se dice que $a$ es \textit{divisor de 0}.
\end{definition}
Existen divisores de 0 a izquierda que no lo son a derecha. Por ejemplo, consideramos el siguiente anillo\[
R=\left\{\begin{pmatrix}
a&b\\ 0 & c
\end{pmatrix}\;\Big|\; a,c\in \Z,\; b\in \Z/2\Z \right\}.
\]Los elementos\[
A=\begin{pmatrix}
    2&0\\0& 1
\end{pmatrix},\quad B=\begin{pmatrix}
    0&\overline 1\\0&0
\end{pmatrix}
\]satisfacen $A\cdot B=0$, por lo que $A$ es divisor de 0 a izquierda. Sin embargo, para cualquier elemento $X$ de $R$,\[
X=\begin{pmatrix}
    x& y\\0&z
\end{pmatrix},
\]se tiene\[
XA=\begin{pmatrix}
    2x& 2y\\0&z
\end{pmatrix}.
\]Esta matriz es 0 si y sólo si $2x=2y=z=0$, si y sólo si $X=0$. Por tanto, $A$ no es divisor de 0 a derecha. Por último, $B^2=0$, por lo que es tanto divisor de 0 a izquierda como a derecha.

\begin{definition}
    Sea $R$ un anillo. Un elemento $a\in R$ es \textit{nilpotente} si $a^n=0$ para algún entero positivo $n$. Al conjunto de elementos nilpotentes del anillo se lo denota por $\Nil(R)$, en honor al matemático Nil Ojeda.
\end{definition}

\begin{definition}
    Sea $R$ un anillo con identidad. Un elemento $a\in R$ es \textit{cuasirregular} si $1-a\in U(R)$.
\end{definition}
Todo elemento nilpotente es irregular, pues si $a^n=0$ para algún entero positivo $n$, entonces \[
(1-a)(1+a+\cdots+a^{n-1})=1-a^n=1.
\]
\begin{definition}
    Un \textit{dominio} es un anillo sin divisores de 0. Si el anillo es conmutativo, con identidad, y es un dominio, entonces se lo denomina \textit{dominio de integridad} o \textit{dominio íntegro}.
\end{definition}
\begin{definition}
    Un anillo con identidad en el que $U(R)=R-\{0\}$ se lo denomina \textit{anillo de división}. Un \textit{cuerpo} es un anillo de división conmutativo.
\end{definition}
\begin{definition}
    Si $R$ es un anillo, el \textit{anillo opuesto}, $R^{op}$, es la terna $(R,+_{op},\cdot_{op})$ con las operaciones\[
    a+_{op}b=a+b,\qquad a\cdot_{op}b=b\cdot a.
    \]
\end{definition}
 
\section{Subanillos e ideales}
\begin{definition}
    Sea $R$ un anillo. Un subconjunto $S\subseteq R$ es un \textit{subanillo} si $S$ es un anillo con las operaciones de $R$.
\end{definition}
Veamos algunos ejemplos:\begin{enumerate}
    \item Si $R$ es un anillo, \[
    \left\{r\in R:\;\forall x\in R,º; rx=xr\right\}
    \]es un subanillo de $R$. Se lo denota por $Z(R)$.
    \item Sea $R$ es un anillo con identidad, y sea $S$ es un anillo tal que $R$ es subanillo de $S$ y $1_R=1_S$. Si $\alpha_1,\dots,\alpha_t\in S$, entonces definimos\[
    R[\alpha_1,\dots,\alpha_t]:=\{f(\alpha_1,\dots,\alpha_t ):\; f\in R[x_1,\dots,x_t]\},
    \]que es un subanillo de $S$. de hecho, es el menor subanillo de $S$ que contiene a $R$ y a $\{\alpha_1,\dots,\alpha_t\}$.
\end{enumerate}
\begin{definition}
    Sea $R$ un anillo. Un subanillo $I\subseteq R$ se dice\begin{enumerate}
        \item \textit{i-ideal} de $R$ si para todo $a\in I$ y para todo $r\in R$, $ra\in I$.
        \item \textit{d-ideal} de $R$ si para todo $a\in I$ y para todo $r\in R$, $ar\in I$.
        \item \textit{ideal} de $R$ si es i-ideal y d-ideal.
    \end{enumerate}
\end{definition}

Veamos algunos ejemplos:\begin{enumerate}
    \item El 0 y $R$ son siempre ideales de $R$. Además, si estos son sus únicos ideals y $R$ tiene identidad, se dice que $R$ es \textit{simple}.
    \item Los ideales de $\Z$ son los subgrupos de $(\Z,+)$.
    \item Sea $R$ anillo con identidad, $S=\textrm M_n(R)$, $n\geq 1$. Entonces\[
    A(d,i):=\{(a_{kl}):\; a_{kl}=0\text{ si }k\neq i\}
    \]es un d-ideal pero no un i-ideal de $S$. La ``d'' viene de ``derecha''. Análogamente, \[
    A(i,j):=\{(a_{kl}):\; a_{kl}=0\text{ si }l\neq j\}
    \]es un i-ideal pero no un d-ideal
\end{enumerate}
\begin{prop}
    Sea $R$ un anillo con identidad e $I$ un d-ideal (o i-ideal, o ideal) de $R$. Entonces $I  =R$ si y sólo si $I\cap U(R)\neq \emptyset$.
\end{prop}
\noindent\textit{Demostración.} Sólo haremos el caso en el que $I$ es i-ideal. Si $I=R$, entonces $1\in I\cap U(R)$. Recíprocamente, si $I\cap U(R)\neq \emptyset$, entonces dado $u\in I\cap U(R)$ se tiene $u\cdot u^{-1}=1\in I$. Por tanto, para todo $r\in R$, $r=r\cdot 1\in I$, es decir, $I=R$. $\qed$

\paragraph{Nota:}Si $R$ tiene identidad y $a\in R$, $aR$ es un d-ideal de $R$, $Ra$ es un i-ideaal de $R$, y $a\in aR\cap Ra$.

\begin{theorem}[Le mola al adolfo]
    Sea $R$ anillo con identidad. Entonces $R$ es anillo de división si y sólo si sus únicos d-ideales (resp. i-ideales) son $0,R$.
\end{theorem}
\noindent\textit{Demostración.} Supongamos que $R$ es un anillo de división. Sea $I$ d-ideal, $I\neq 0$. Entonces existe $0\neq r\in I$, y por tanto $r\cdot r^{-1}=1\in I$. Por tanto, $I=R$. Recíprocamente, si $0\neq r\in R$, entonces $0\neq rR$, siendo $rR$ un d-ideal de $R$. Por tanto, $rR=R$, y por tanto existe $s\in R$ tal que $rs=1$, es decir, $r$ es unidad. Por tanto $R$ es anillo de división.$\qed$

\begin{corollary}
    Sea $R$ anillo conmutativo con identidad. Entonces $R$ es cuerpo si y sólo si $R$ es anillo simple.
\end{corollary}
La hipótesis de conmutatividad es necesaria, pues si consideramos $R=\mathrm M_n (D)$ para algún anillo de división $D$, se tiene que $R$ es simple pero no es un anillo de división porque $A(d,i), A(i,j)$ son d-ideales o i-ideales distintos de 0 y del total.

\begin{definition}
    Sea $R$ anillo, $X\subseteq R$, $I\subseteq R$ ideal. Entonces:\begin{enumerate}
        \item El siguiente conjunto es un i-ideal de $R$\[
        IX\coloneq \left\{\sum_{i=1}^n a_ix_i:\; a_i\in I,\; x_i\in X,\; n\in \N\right\}
        \]
        \item El siguiente conjunto es un d-ideal de $R$\[
            IX\coloneq \left\{\sum_{i=1}^n x_ia_i :\; a_i\in I,\; x_i\in X,\; n\in \N\right\}
        \]
    \end{enumerate}
\end{definition}
\begin{definition}
    Sea $R$ anillo y $X\subseteq R$. El d-ideal generado por $X$ es la intersección de los d-ideales de $R$ que contienen a $X$, y es denotado por $(X)_d$. Análogamente se define el i-ideal generado por $X$, denotado $(X)_i$, y el ideal generado por $X$, $(X)$. Si $X=\emptyset$, el d-ideal, el i-ideal y el ideal generado por $X$ se define como el conjunto vacío.
\end{definition}
\begin{theorem}
    Sea $X\neq \emptyset$. Sea $R$ anillo con identidad. Entonces\begin{enumerate}
        \item $(X)_d=XR.$
        \item $(X)_i=RX.$
        \item $(X)=RXR$.
    \end{enumerate}
\end{theorem}
\noindent\textit{Demostración.}Probamos 3. En primer lugar, notamos que $X\subseteq RXR$, ya que si $x\in X$, $x=1\cdot x\cdot 1\in RXR$. Además, es fácil ver que\[
RXR=\left\{\sum_{i=1}^n a_ix_ib_i:\; a_i\in I,\; b_i\in I,\; n\in \N\right\}
\]es ideal de $R$. Por último, si $I\subseteq R$ es un ideal tal que $X\subseteq I$, entonces cualquier elemento de $RXR$ es de la forma\[
    \sum_{i=1}^n a_ix_ib_i,\quad a_i\in R,\; b_i\in R,\; n\in \N
\]y como $x_i\in I$ para todo $i\in\{1,\dots,n\}$, se tiene que $a_ix_i\in I$ y por tanto $a_ix_ib_i\in I$, por ser $I$ ideal. Así,\[
    \sum_{i=1}^n a_ix_ib_i\in I.
\]Y por tanto $RXR\subseteq I$. Es decir, $RXR$ es el menor ideal de $R$ que contiene a $X$. Concluimos así que $(X)=RXR$. $\qed$

\paragraph{Nota:}Si $X=\{a_1,\dots,a_t\}\subseteq R$, entonces\[
\begin{array}{ll}
    1. & (X)_i=\left\{\sum_{i=1}^t r_ia_i:\; r_i\in R,\; 1\leq i\leq t\right\}\\
    2. & (X)_d=\left\{\sum_{i=1}^t a_is_j:\; s_i\in R,\; 1\leq i\leq t\right\}\\
    3. & (X)=\left\{\sum_{i=1}^t r_ia_is_j:\; r_i,s_j\in R,\; 1\leq i\leq t\right\}
\end{array}
\]
\begin{definition}
    Un i-ideal (d-ideal, ideal) de $R$ es \textit{finitamente generado}, abreviado \textit{f.g.}, si existen elementos $a_1,\dots,a_t\in R$, $t\in \N$, tales que\[
    I=(\{a_1,\dots,a_t\})\eqqcolon(a_1,\dots,a_t)
    \]Si $t=1$, $I$ es un \textit{ideal principal} de $R$. En este caso, y si $R$ tiene identidad, $I=(a)=RaR$, $(a)_i=Ra$, $(a)_d=aR$.
\end{definition}
\begin{definition}
    Un anillo con identidad es \textit{anillo principal} si todo ideal es principal. Si, además, es un dominio, se lo denomina \textit{dominio de ideales principales}, abreviado DIP.
\end{definition}Veamos algunos ejemplos.\begin{enumerate}
    \item $\Z$ es un DIP.
    \item Si $R$ es un dominio de integridad, $R[x]$ es un DIP si y sólo si $R$ es cuerpo (ejercicio).
    \item Sea $R=\R^{[0,1]}$. Sea $a\in[0,1]$. El subconjunto\[
    R_a\coloneq \{f\in R:\; f(a)=0\}
    \]es un ideal principal. Sea\[
    \begin{array}{rlcl}
        \ell:&[0,1]&\longrightarrow &\R\\
        & x & \longmapsto & 1\text{ si }x\neq a\\
        & x & \longmapsto & 0\text{ si }x= a,\\
    \end{array}
    \]de modo que $\ell\in R_a$. Por tanto $(\ell)$ es un subideal de $R_a$. Por otro lado, dado $f\in R_a$, se tiene que $f=f\ell$. Por tanto, $f\in(\ell)$. Concluimos así que $R_a=(\ell)$ es un ideal principal.
    \item Sea $S=\{f:[0,1]\longrightarrow \R:\; f\text{ es continua}\}$. Es un subanillo de $\R^{[0,1]}$. Sea $a\in[0,1]$. Definimos $S_a=\{f\in S:\; f(a)=0\}$, que es un ideal de $S$. Este ideal no es finitamente generado (ejercicio).
    \item Sea $K$ un cuerpo, y consideramos la cadena de anillos\[
    K[x_1]\subseteq K[x_1,x_2]\subseteq\cdots.
    \]Definimos\[
    K[x_1,x_2,\dots]\coloneq\bigcup_{i\geq 1}K[x_1,\dots,x_i].
    \]Este es un anillo conmutativo con identidad. El ideal $I$ generado por todas las indeterminadas no es finitamente generado.
\end{enumerate}

\section{Anillo cociente}
Sea $R$ un anillo, $I\subseteq R$ un ideal. Entonces $(I,+)$ es un subgrupo normal de $(R,+)$, por lo que podemos considerar el cociente $R/I$ como grupo. A este grupo le añadimos una operación $\cdot$ dada por\[
(a+I)\cdot (b+I)=ab+I,\quad a,b\in R.
\]Comprobamos que $\cdot$ es aplicación. Si $a+I=a_1+I$ y $b+I=b_1+I$, el hecho de que $I$ es ideal garantiza lo siguiente\[
ab-a_1b_1=ab-a_1b+a_1b-a_1b_1=(a-a_1)b+a_1(b-b_1)\in I.
\]La terna $(R/I,+,\cdot)$ es el anillo denominado ``anillo cociente'' de $R$ por $I$.
\subsection{Homomorfismo de anillos}
\begin{definition}
    Sean $R_1,R_2$ anillos. Una aplicación $f:R_1\longrightarrow R_2$ es \textit{homomorfismo de anillos}, abreviado \textit{HM} de anillos, si\[
    f:(R_1,+)\longrightarrow (R_2,+)
    \]es homomorfismo de grupos y además $f(xy)=f(x)f(y)$ para todo $x,y\in R_1$.
\end{definition}Si $f:R_1\longrightarrow R_2$ es un HM de anillos, de la definición se derivan las siguientes propiedades:\begin{enumerate}
    \item  $f(0_{R_1})=0_{R_2}$.
    \item $f(-a)=-f(a)$.
    \item Si $R_1$ y $R_2$ tienen identidad, en general $f(1_{R_1})\neq 1_{R_2}$. Basta considerar la aplicación constante 0, que es HM de anillos. Sin embargo, si $f$ es sobreyectiva, entonces $f(1_{R_1})=1_{R_2}$.
\end{enumerate}

\begin{theorem}
    Sea $f:R_1\longrightarrow R_2$ un HM de anillos. Entonces\begin{enumerate}
        \item Si $A\subseteq R_1$ es un subanillo, entonces $f(A)$ es un subanillo de $R_2$. En particular,\[
        \Im(f)\coloneq f(R_1)\subseteq R_2
        \]es subanillo de $R_2$.
        \item Si $I\subseteq R_1$ es un ideal, entonces no necesariamente $f(I)$ es un ideal de $R_2$. Sin embargo, si $f$ es sobreyectiva, entonces $f(I)$ es un ideal de $R_2$.
        \item Si $A$ es un subanillo (resp. ideal) de $R_2$, entonces $f^{-1}(A)$ es un subanillo (resp. ideal) de $R_1$. En particular,\[
        \ker(f)\coloneq f^{-1}(0_{R_2})\subseteq R_1
        \]es un ideal de $R_1$ contenido en cada $f^{-1}(J)$, $J$ ideal de $R_2$.
        \item Se cumple el \textit{teorema de isomofía} para anillos\[
        R_1/\ker(f)\cong \Im(f).
        \]
    \end{enumerate}
\end{theorem}
\noindent\textit{Demostración.} \begin{itemize}
    \item[1.] Sean $r,s\in f(A)$, donde $A$ es un subanillo de $R_1$. Entonces $r=f(a),s=f(b)$ para algunos $a,b\in R_1$. Por tanto\[
    rs=f(a)f(b)=f(ab)\in f(A),\quad r+s=f(a)+f(b)=f(a+b)\in f(A),
    \]concluyendo que $f(A)$ es un subanillo de $R_2$. En particular, $\Im(f)$ es subanillo de $R_2$.
    \item[2.] Como contraejemplo consideramos el ideal $n\Z$ de $\Z$ y la aplicación inclusión $i:\Z\longrightarrow\Q$. El conjunto $i(n\Z)$ no es un ideal de $\Q$ para ningún $n$ entero positivo. 
    
    Supongamos que $f$ es sobreyectiva. Veamos que $f(I)$ sí es un ideal de $R_2$. Sea $r\in R_2$, $a\in f(I)$. Dado que $f$ es sobreyectiva existe $s\in R_1$ tal que $f(s)=r$. Sea $i\in I$ tal que $a=f(i)$. Entonces $ra=f(s)f(i)=f(si)\in f(I)$, dado que al ser $I$ ideal, $si\in I$.
    \item[3.] Supongamos que $A$ es un subanillo de $R_2$. Sean $r,s\in f^{-1}(A)$. Entonces $f(r),f(s)\in A$. Como $A$ es subanillo, $f(r)+f(s),f(r)f(s)\in A$. Como $f$ es homomorfismo, $f(r+s),f(rs)\in A$. Concluimos que $r+s,rs\in f^{-1}(A)$, luego $f^{-1}(A)$ es subanillo. Además, si $J$ es un ideal de $R_2$, dado que $0\in J$ se tiene que $\ker f=f^{-1}(\{0\})\subseteq f^{-1}(J)$.
    
    \item[4.] Son isomorfos como grupos. Consideramos \[\begin{array}{rlcl}
        \overline f:&R_1/\ker(f)&\longrightarrow &\Im(f)\\
        &x+\ker(f)&\longmapsto &f(x).
    \end{array}\]
    Entonces\[
    \overline f[(x + \ker(f))(y + \ker(f))]=\overline f[xy+\ker(f)]=f(xy)=f(x)f(y)=\overline f(x+\ker(f))\overline f(y+\ker(f)),
    \]concluyendo así que $\overline f$ es homomorfismo de anillos.$\qed$
\end{itemize}Veamos algunos ejemplos\begin{enumerate}
    \item Sea $I$ un ideal de $R$. La aplicación $\rho:R\longrightarrow R/I$ es un epimorfismo que satisface $\ker\rho=I$.
    \item Sea $I$ un ideal de $R$. La aplicación $\overline\rho:\mathrm M_n(R)\longrightarrow \mathrm M_n(R/I)$ dada por\[
    \overline\rho[(a_{ij})]=(\rho(a_{ij}))
    \]es un epimorfismo de anillos que satisface $\ker\overline\rho=\mathrm M_n(I)$.$\qed$
\end{enumerate}
A continuación nos disponemos a mostrar qué forma tienen los ideales del anillo cociente. Sea $R$ un anillo, $I\subseteq R$ un ideal. Sea $J$ cualquier otro ideal de $R$. Dado que $\rho$ es un epimorfismo, $\rho(J)$ es un ideal de $R/I$.\[
\rho(J)=\{j+I:\; j\in J\}=\{j+i+I:\; j\in J,i\in I\}.
\]Consideramos el siguiente ideal de $R$\[
I+J\coloneq\{i+j:\; i\in I,j\in J\}.
\]Este ideal satisface $I\subseteq I+J$. Así, $I$ es ideal de $I+J$. Por tanto,\[
\rho(J)=(I+J)/I
\]es un ideal de $R/I$. Reciprocamente, si $A$ es un ideal de $R/I$, tomamos el ideal de $R$\[
B\coloneq \rho^{-1}(A),
\]que satisface $\ker\rho =I \subseteq B$. Al ser $\rho$ epimorfismo, \[
A=\rho(\rho^{-1}(A))=\rho(B)=(I+B)/I=B/I.
\]Por último, si se da la igualdad\[
A_1/I=A_2/I
\]para ideales $A_1,A_2$ de $R$, entonces\[
A_1=\rho^{-1}(A_1/I)=\rho^{-1}(A_2/I)=A_2.
\]Esto nos permite concluir lo siguiente
\begin{theorem}
    Los ideales del anillo cociente $R/I$ son los cocientes de la forma $J/I$ donde $J\subseteq R$ es un ideal que contiene a $I$.
\end{theorem}
\begin{theorem}[Segundo teorema de isomorfía]
    Sea $R$ un anillo, $I,J\subseteq R$ ideales. Entonces\[
    J/(I\cap J)\cong (I+J)/I.
    \]
\end{theorem}
\noindent\textit{Demostración.} Consideramos el epimorfismo\[
\overline\rho=\rho|_J:J\longrightarrow \rho(J)=(I+J)/I.
\]con núcleo $\ker\overline\rho = I\cap J$. Por el primer teorema de isomorfía,\[
J/(I\cap J)\cong (I+J)/I.
\]$\qed$
\begin{theorem}[Tercer teorema de isomorfía]
    Sea $R$ un anillo, $I\subseteq J\subseteq R$ ideales tales. Entonces\[
    (R/I)/(J/I)\cong R/J.
    \]
\end{theorem}
\noindent\textit{Demostración.} Consideramos la correspondencia\[
\begin{array}{rlcl}
    \varphi:&R/I&\longrightarrow &R/J\\
    &r+I&\longmapsto &r+J.
\end{array}
\]Esta correspondencia es una aplicación, pues\[
r+ I=r_1+I\implies r-r_1\in I\subseteq J\implies \varphi(r) = r+J=r_1+J = \varphi(r_1).
\]Es más, $\varphi$ es un epimorfismo de anillos con núcleo $\ker\varphi=J/I$. Por el primer teorema de isomorfía,\[
(R/I)/(J/I)\cong R/J.
\]$\qed$
\section{Operaciones con ideales.}
\begin{definition}
    Sea $\{L_i\}_{i\in I}$ una familia de ideales de un anillo $R$. Entonces\[
    \sum_{i\in I}L_i\coloneq\left(\bigcup_{i\in I} L_i\right).
    \]Si $R$ tiene identidad, entonces\[
    \sum_{i\in I}L_i=\left\{\sum_{j=1}^t l_j:\; l_j\in L_{i_j},\; i_j\in I,\; 1\leq j\leq t\in\N\right\}
    \]Se dice que la suma es \textit{directa} si para todo índice $i\in I$, se cumple\[
    L_i\cap\left(\sum_{j\neq i}L_j\right)=0
    \]En este caso, denotamos a la suma por\[
    \bigoplus_{i\in I}L_i.
    \]
\end{definition}
\begin{prop}
    Si $\{L_i\}_{i\in I}$ es una familia de ideales de un anillo $R$ cuya suma es directa, para todo elemento $x$ de la suma existen índices únicos $i_1,\dots,i_t\in I$ y elementos no nulos únicos $x_1\in L_{i_1},\dots,x_t\in L_{i_t}$ tales que\[
    x=x_1+\cdots+x_t.
    \]Además, si la suma es finita\[
    L_1\oplus L_2\oplus \cdots\oplus L_n\cong L_1\times L_2\times\cdots\times L_n.
    \]
\end{prop}
\noindent\textit{Demostración.} Veamos primero que el cero no se puede expresar como suma de elementos no nulos. Por reducción al absurdo, si $0=x_1+\dots+x_n$, para $0\neq x_i$ en sus respectivos $L_{j_i}$, todos los $L_{j_i}$ distintos dos a dos, entonces\[
x_1=-x_2-\dots-x_n\in L_{i_1}\cap\left(\sum_{j\neq i_1}L_{i_j}\right)=0.
\]lo que supone una contradicción.

Sea $x\in\oplus_{i\in I}L_i$. Supongamos que existen $x_1,\dots,x_t,y_1,\dots,y_s$ elementos no nulos de $L_{i_1},\dots,L_{i_t},L_{j_1},\dots,L_{j_s}$ respectivamente tales que\[
x=x_1+\cdots+x_t=y_1+\cdots+y_s.
\]Supongamos que $t>s$. Entonces\[
    x_1+\cdots+x_t-y_1-\cdots-y_s=0.
\]Y esto implica necesariamente que alguno de los $x_i$ ha de ser 0, en contradicción con lo supuesto. Análogamente se prueba que $t<s$ no es posible. Por tanto, $t=s$.
Supongamos ahora que existe algún $L_{i_k}$ que es distinto a todos los $L_{j_l}$, $1\leq k,l\leq t$. Entonces podríamos expresar $x_{i_k}$ como suma de elementos que no están en $L_{i_k}$, lo que implica $x_{i_k}=0$. Esta contradicción nos dice que podemos reordenar los elementos $x_i,y_j$ de forma que $x_i,y_i\in L_i$ para todo $i\in{1,\dots, t}$. Ahora, tenemos\[
(x_1-y_1)+\dots + (x_t-y_t)=0,
\]y por tanto cada término es 0. Esto demustra que la expresión es única.$\qed$
\begin{definition}
    Sean $L_1,\dots,L_n\subseteq R$ ideales. El producto $L_1\cdot\cdots\cdot L_n$ es el ideal generado por el conjunto\[
    \{x_1\cdots x_n:\; x_i\in L_i,\; 1\leq i\leq n\}.
    \]En concreto:\[
    L_1\cdots L_n=\left\{\prod_{i=1}^n l_i:\; l_i\in L_i \right\}
    \]Además, si $I$ es un ideal de $R$, se define $I^1\coloneq I$, $I^{n+1}\coloneq I^n\cdot I$.
\end{definition}
\begin{definition}
    Se dice que un ideal $I$ de un anillo $R$ es \textit{nilpotente} si $I^n=0$ para algún entero positivo $n$.
\end{definition}
Por ejemplo, si $I\subseteq R$ es un ideal, para cualquier entero positivo $n$, $I/I^n$ es ideal nilpotente. Esto es fácil de comprobar.

\begin{theorem}
    Sea $R$ un anillo con identidad. Entonces su caracteristica es un número primo o 0.
\end{theorem}
\noindent\textit{Demostración.} Considérese el homomorfismo\[
\begin{array}{rlcl}
    \varphi:&\Z &\longrightarrow &R\\
    &n &\longmapsto &n\cdot 1_R,
\end{array}
\]cuyo núcleo es un ideal de $\Z$. Por ser $\Z$ un dominio de ideales principales, $\ker\varphi=(t)$, para algún $t\in\Z$. Si $t=0$, para todo $n\in\Z$, $n\cdot 1_R=0$, y por tanto la característica de $R$ es 0. Si $t\neq 0$, entonces $t$ es un número primo, y por tanto la característica de $R$ es $t$.$\qed$

\begin{lemma}[Binomio de Newton]
    Sea $R$ un anillo conmutativo con identidad. Entonces\[
    (a+b)^n=\sum_{k=0}^n\binom{n}{k}a^kb^{n-k}
    \]
\end{lemma}

\begin{prop}
    Sea $R$ un anillo conmutativo con identidad. Entonces \[
    \Nil(R)=\{x\in R:\; x\text{ es nilpotente}\}
    \]es ideal de $R$.
\end{prop}
\noindent\textit{Demostración.} Sea $x\in \Nil(R)$, $r\in R$. Entonces\[
(rx)^n=r^nx^n=0,
\]y por tanto $rx\in \Nil(R)$. Además, si $a,b\in\Nil(R)$, existen enteros positivos $n,m$ tales que\[
a^n=b^m=0.
\]Por el binomio de Newton,\[
(a+b)^{n+m}=0,
\]y por tanto $a+b\in\Nil(R)$.$\qed$

Si un ideal es nilpotente, todos sus elementos son nilpotentes. El recíproco no es cierto. Por ejemplo, si $K$ es un cuerpo, tomamos\[
S\coloneq K[x_1,x_2,\dots],\quad J\coloneq K[x_1^2,x_2^3,\dots]\subseteq (x_1,x_2,\dots)\eqqcolon I,\quad R\coloneq S/J
\]Todo elemento de $I/J$ es nilpotente: si $x_l\in I$ entonces $x_l^{l+1}\in J$, luego\[
x_l + J\in\Nil(R)
\]Sin embargo, el ideal $I/J$ no es nilpotente (ejercicio).

\section{Ideales primos y maximales}
A partir de aquí, todo anillo será considerado con identidad, aunque no se diga, salvo que se indique lo contrario.

\begin{definition}
    Un ideal $P$ de un anillo $R$ se dice \textit{primo} si verifica las siguientes condiciones:\begin{enumerate}
        \item $P\neq R$.
        \item Si $I,J$ son ideales de $R$ tales que $IJ\subseteq P$, entonces $I\in P$ o $J\in P$.
        \end{enumerate}
\end{definition}


\paragraph{Notación:}\[
\Spec(R)\coloneq\{P|\; P\text{ ideal primo de }R\}.
\]Si $D$ es un anillo de división, $\Spec(\mathrm M_n(D))=\{0\}$, ya que los ideales de $\mathrm M_n(D)$ son $\{0,\mathrm M_n(D)\}$.

\begin{prop}
    Sea $R$ un anillo y sea $P\subsetneq R$ un ideal. Entonces $P$ es primo si y sólo si para todo $a,b$ en $R$ tales que $ab\in P$, se tiene $a\in P$ o $b\in P$.
\end{prop}

\begin{prop}
    Sea $R$ anillo con identidad y $P\subsetneq R$ un ideal. Las siguientes afirmaciones son equivalentes:\begin{enumerate}
        \item $P$ es primo.
        \item Si $a,b\in R$ son tales que $(a)(b)\subseteq P$ entonces $(a)\subseteq P$ o $(b)\subseteq P$.
        \item Si $a,b\in R$ son tales que $aRb\in P$ entonces $a\in P$ o $b\in P$. [Importante]
        \item Si $A,B$ son i-ideales de $R$ tales que $AB\subseteq P$, entonces $A\subseteq P$ o $B\subseteq P$.
        \item Si $A,B$ son d-ideales de $R$ tales que $AB\subseteq P$, entonces $A\subseteq P$ o $B\subseteq P$.
    \end{enumerate}
\end{prop}
\noindent\textit{Demostración.} 1 implica 2 trivialmente. Supongamos que 2 se cumple.

Si $aRb\subseteq P$, entonces\[
(RaR)(RbR)\subseteq RPR=P,
\]y por tanto\[
(a)(b)\subseteq P,
\]y por la condición 2, $a\in P$ o $b\in P$.

Supongamos que se satisface la condición 3. Sean $A,B$ i-ideales de $R$ tales que $AB\subseteq P$. Supongamos $A\subsetneq P$, y sea $a\in A-P$. Sea $b\in B$. Como $A$ es i-ideal, $Ra\subseteq A$. Análogamente, $Rb\subseteq B$. Por tanto,\[
aRb\subseteq (Ra)(Rb)\subseteq AB\subseteq P.
\]Así, $aRb\subseteq P$, y por la condición 3, $a\in P$ o $b\in P$. Como $a\notin P$, necesariamente $b\in P$. Concluimos que $B\subseteq P$.

Veamos que 4 implica 5. Sean $A,B$ d-ideales de $R$ tales que $AB\subseteq P$. Entonces\[
R\underbrace{AR}_AB\subseteq RP=P.
\]Como $RA,RB$ son i-ideales, se tiene que $RA\subseteq P$ o $RB\subseteq P$. Por tanto, $A\subseteq P$ o $B\subseteq P$.

Todo ideal es d-ideal, por lo que 5 implica 1.$\qed$

\paragraph{Ejercicio:} Sean $A$ un i-ideal, $B$ es d-ideal tales que $AB\subseteq P$. Esto no es condición suficiente para asegurar $A\subseteq P$ o $B\subseteq P$.

\begin{prop}
    Sea $R$ un anillo conmutativo con identidad, y sea $P\neq R$ un ideal. El ideal $P$ es primo si y sólo si $R/P$ es un dominio de integridad.
\end{prop}
\noindent\textit{Demostración.} Si $P$ es primo, entonces $P\neq R$, luego $1\notin P$, luego\[
1+P\neq 0+ P.
\]Así, $R/P$ es un dominio. Si $[a],[b]$ son elementos de $R/P$ cuyo producto es cero, entonces\[
[ab]=[a][b]=0.
\]Por tanto, $ab\in P$. Como $P$ es primo, o bien $a$ está en $P$, o $b$ está en $P$. Por tanto, $[a]=0$ o $[b]=0$, y por tanto $R/P$ es un dominio.

Recíprocamente, si $A/P$ es un dominio íntegro, necesariamente\[
1+P\neq 0+ P,
\]por lo que $1\notin P$, luego $P\neq R$. Así, si $a,b$ son elementos de $R$ tales que $ab\in P$, entonces\[
(a+P)(b+P)=(ab+P)=0,
\]y como $R/P$ es dominio íntegro, necesariamente $(a+P)=0+P$ o $(b+P)=0+P$, es decir, $a\in P$ o $b\in P$. Por tanto, $P$ es primo.$\qed$

\paragraph{Ejemplo:} Sea $R=K[x,y]$, con $K$ cuerpo. Usaremos la proposición anterior para demostrar que $(x),(y)$ son ideales primos de $R$. Consideramos el epimorfismo de anillos\[
\begin{array}{rlcl}
    \varphi:&K[x,y]&\longrightarrow &K[x]\\
    &f(x,y)&\longmapsto &f(x,0).
\end{array}
\]Un elemento $f\in K[x,y]$, que podemos escribir de la forma\[
f(x,y)=a_0(x)+\sum_{i=1}^n a_i(x)y^i,
\]pertenece al núcleo de $\varphi$ si y sólo si\[
f(x,0)=a_0(x)=0.
\]Por tanto,\[
\ker\varphi = (y).
\]Por otro lado, al ser $\varphi$ epimorfismo, el primer teorema de isomorfía nos dice que\[
K[x,y]/(y)\cong K[x],
\]y dado que $K[x]$ es dominio íntegro, $(y)$ es ideal primo en $K[x,y]$.

\section{Subconjuntos multiplicativos y m-conjuntos.}
\begin{definition}
    Seaa $R$ un anillo, $\emptyset\neq S\subseteq R$. Entonces\begin{enumerate}
        \item $S$ es \textit{multiplicativo} si $1\in S$ y si $a,b\in S$ entonces $ab\in S$.
        \item $S$ es un m-conjunto si $a,b\in S$ implica la existencia de un elemento $r\in R$ tal que $arb\in S$.
    \end{enumerate}
\end{definition}
Veamos algunos ejemlos:\begin{enumerate}
    \item Si $S$ es un subconjunto multiplicativo de $R$, entonces $S$ es un m-conjunto. El recíprono no es cierto, como veremos en el siguiente ejemplo.
    \item $S\coloneq\{1,a,a^2,a^4,a^8,\dots\}$ es un m-conjunto que no es multiplicativo, ya que $a\cdot a^2=a^3\notin S$.
    \item Si $P$ es un ideal primo de $R$, entonces $R-P$ es un m-conjunto. La siguiente proposición muestra que el recíproco es cierto.
\end{enumerate}
\begin{prop}
    Sea $P\neq R$ ideal. Entonces $P$ es primo si y sólo si $S=R-P$ es un m-conjunto.
\end{prop}
\noindent\textit{Demostración.} Supongamos que $P$ es primo, y $a,b\in S$. Supongamos por reducción al absurdo que $a,b\notin P$. En este caso, $aRb\subsetneq P$. Sea $r\in R$ tal que $arb\notin P$. Entonces\[
arb\in R-P=S
\]concluimos que $S$ es un m-conjunto.

Recíprocamente, si $S$ es un m-conjunto, sean $a,b\in R$ tales que $aRb\subseteq P$. Supongamos por reducción al absurdo que $a,b\notin P$. Entonces $arb\in S$ para algún elemento $r$ de $R$. Así,\[
arb\in S\cap aRb\subseteq S\cap P=\emptyset,
\]lo cual es una contradicción.
faf
\begin{lemma}\label{lema:condicion-primalidad-m-conjunto}
    Sea $S$ un m-conjunto de $R$. Sea $P\subsetneq R$ un ideal maximal respecto a la condición $S\cap P=\emptyset$. Entonces $P$ es un ideal primo.
\end{lemma}
\noindent\textit{Demostración.} Sean $a,b\in R$ tales que $(a)(b)\in P$. Supongamos por reducción al absurdo que $(a)\nsubseteq P$ y $(b)\subsetneq P$. Entonces $P\subsetneq P+(a)$ y $P\subsetneq P+(b)$. Por tanto\[
S\cap (P+(a))\neq \emptyset\neq S\cap (P+(b)).
\]Por tanto, existe $s_i=p_i+\lambda_i\in S$ con $p_i\in P$, $\lambda_1\in(a)$, $\lambda_2\in (b)$, $i=1,2$. Dado que $S$ es un m-conjunto, existe un elemento $r$ de $R$ tal que $s_1rs_2\in S$. Ahora bien,\[
s_1rs_2=(p_1+\lambda_1)r(p_2+\lambda_2)\in (P+(a))(P+(b))\subseteq P+(a)(b)\subseteq P.
\]Por tanto, $s_1rs_2\in S\cap P=\emptyset$, lo cual es una contradicción.

\begin{definition}
    Sea $I\subsetneq R$ un ideal. Definimos el radical de $I$, $\Rad(I)$, como\[
    \Rad(I)=\{x\in R:\; x\in S\subseteq R,\, S\text{ m-conjunto}\implies S\cap I=\emptyset\}.
    \]
\end{definition}
\begin{prop}
    Sea $I\subsetneq R$ un ideal. Entonces\[
    \Rad(I)\subseteq\{x\in R:\;\exists x^n=1\text{ para algún }n\geq 1\}.
    \]Si $R$ es conmutativo, se da la igualdad.
\end{prop}
\noindent\textit{Demostración.} Sea $x\in \Rad(I)$. El conjunto \[
S_x=\{1,x,x^2,\dots\}
\]es multiplicativo, y por tanto un m-conjunto. Además, dado que $x\in S$, se tiene que $S\cap I\neq \emptyset$. Sea $y\in S\cap I$. Entonces $y=x^n$ para algún $n\geq 1$. 

Supongamos que $R$ es conmutativo. Sea $x\in R$ tal que $x^n=I$ para algún $n\geq 1$. Sea $S$ un m-conjunto de $R$ tal que $x\in S$. Entonces existe $r_1\in S$ tal que $xr_1x=x^2r_1\in S$. Por inducción se prueba que existe $r_{n-1}$ tal que $x^nr_{n-1}\in S$. Por tanto $x^nr_{n-1}\in S\cap I\neq \emptyset$. Deducimos así que $x$ pertenede a $\Rad(I)$.

\begin{prop}
    Sea $R$ un anillo conmutativo e $I\subseteq R$ un ideal. Entonces:\begin{enumerate}
        \item $\Rad(0)=\Nil(R)\subseteq R$ es un ideal.
        \item $\Rad(I)$ es un ideal de $R$.
        \item $I\subseteq \Rad(I)$.
        \item $\Rad(I)/I=\Nil(R/I)$.
    \end{enumerate}
\end{prop}
\noindent\textit{Demostración.}\begin{enumerate}
    \item Se deduce de la proposición anterior.
    \item Sean $x,y\in \Rad(I)$. Entonces existen $n,m\geq 1$ tales que $x^n,y^m\in I$. Así,\[
    (x+y)^{n+m}=\sum_{k=0}^{n+m}\binom{n+m}{k}x^ky^{m+n-k}\in I.
    \]Por tanto $x+y\in \Rad(I)$. Además, si $r\in R$, entonces $(rx)^n=r^nx^n\in I$. Por tanto $rx\in \Rad(I)$. Concluimos así que $\Rad(I)$ es un ideal.
    \item Dado $x\in I$, se tiene $x^n\in I$ para $n=1$, luego $x\in\Rad(I)$.
    \item Sea $x+I\in \Rad(I)/I$. Entonces existe $n\geq 1$ tal que $x^n\in I$. Por tanto,\[
    (x+I)^n=0 + I.
    \]Por tanto, $x+I\in\Nil(R/I)$. Recíprocamente, sea $x+I\in\Nil(R/I)$. Entonce existe $n\geq 1$ tal que \[
    (x+I)^n=0+I,
    \]y por tanto $x^n\in I$, luego $x\in\Rad(I)$. Así, $x+I\in \Rad(I)/I$. Concluimos que $\Rad(I)/I=\Nil(R/I)$.
\end{enumerate}

\begin{theorem}
    Sea $R$ un anillo, $I\neq R$ un ideal. Entonces \[
    \Rad(I)=\bigcap_{I\subseteq P\in\Spec(R)}P.
    \]
\end{theorem}
\noindent\textit{Demostración.} Sea $x\in \Rad(I)$. Sea $n\geq 1$ tal que $x^n\in I$. Si $P\in\Spec(R)$ contiene a $I$, entonces $x^n\in P$, y como $P$ es primo, necesariamente $x\in P$. Esto implica la siguiente inclusión:\[
    \Rad(I)\subseteq\bigcap_{I\subseteq P\in\Spec(R)}P.
\]Sea ahora $x\in R$ contenido en todos los ideales primos de $R$ que contienen a $I$. Supongamos por reducción al absurdo que $x\notin \Rad(I)$. Entonces existe un m-conjunto $S$ tal que $S\cap I=\emptyset$. Consideramos el siguiente conjunto con el objetivo de aplicar el lema de Zorn\[
\tau=\{J\neq R\textit{ ideal}:\; J\subseteq I,\,\land\, J\cap S=\emptyset\}.
\]Este conjunto es no vacío porque contiene al elemento $I$. Además, es un conjunto ordenado por inclusión. Sea $\{C_k\}_{k\in K}$ una cadena de $\tau$. Tomamos el conjunto\[
C\coloneq\bigcup_{k\in K}C_k.
\]Veamos que $C\in \tau$. En efecto,\[
C\cap S=\left(\bigcup_{k\in K}C_k\right)\cap S=\bigcup_{k\in K}(C_k\cap S)=\emptyset.
\]y si $c\in C$, entonces $c\in C_k$ para algún $k\in I$, y por tanto $c\in I$. Así, $C\subseteq I$. Vemos además que $C$ es una cota superior de la cadena. Concluimos por el lema de Zorn que existe un elemento maximal de $\tau$. Sea $P$ ese elemento maximal. Por el lema \ref{lema:condicion-primalidad-m-conjunto}, el ideal $P$ es maximal, por lo que $x\in P$. Pero entonces $x\in P\cap S=\emptyset$, lo que supone una contradicción. Concluimos que $x\in\Rad(I)$.$\qed$

\begin{prop}
    Seaa $I\subseteq R$ un ideal. Entonces\begin{enumerate}
        \item Si $I\in\Spec(R)$, entonces $\Rad(I)=I$.
        \item Sea $P\in\Spec(R)$. Entonces $I\subseteq P$ si y sólo si $\Rad(I)\subseteq P$. En particular,\[
        \Rad(\Rad(I))=\Rad(I).
        \]
        \item Si $I,J$ son ideales de $R$, entonces\[
        \Rad(IJ)=\Rad(I\cap J).
        \]En particular, para todo entero positivo $n$,\[
        \Rad(I^n)=\Rad(I).
        \]
    \end{enumerate}
\end{prop}
\noindent\textit{Demostración.}\begin{enumerate}
    \item Se sigue del teorema anterior. Si $I$ no es primo, no tiene por qué cumplirse: basta tomar el ideal $I=(p^n)\subseteq\Z$ para cualquier $n\geq 2$.
    \item Sea $P\in\Spec(R)$ tal que $I\subseteq P$. Entonces\[
    \Rad(I)=\bigcap_{\substack{
        I\subseteq Q\\ Q\in\Spec(R)}}Q\subseteq P.
    \]Recíprocamente, $\Rad(I)\subseteq P$, del hecho de que $I\subseteq\Rad(I)$ se deduce directamente que $I\subseteq P$. En este caso,\[
    \Rad(\Rad(I))=\bigcap_{\substack{\Rad(I)\subseteq P\\ P\in\Spec(R)}}P=\bigcap_{\substack{I\subseteq P\\ P\in\Spec(R)}}P=\Rad(I)
    \]
    \item Sea $P\in \Spec(R)$. Si $L\subseteq M\subseteq R$ son ideales y $M\subseteq P$, entonces $L\subseteq P$. Es decir, hay más primos conteniendo a $L$ que a $M$. Al intersectarlos, obtenemos que\[
    \Rad(L)\subseteq \Rad(M).
    \]Es decir, el radical conserva inclusiones. En particular, dadas las inclusiones\[
    IJ\subseteq I,\quad IJ\subseteq J,\quad IJ\subseteq I\cap J,
    \]se obtienen las inclusiones\[
    \Rad(IJ)\subseteq\Rad(I)\Rad(J),\quad \Rad(IJ)\subseteq \Rad(I\cap J)\subseteq \Rad(I)\cap \Rad(J).
    \]Además, como $P$ es primo, si $IJ\subseteq P$, entonces $I\subseteq P$ o $J\subseteq P$. En cualquier caso, $I\cap J\subseteq P$. Esto implica que\[
    \Rad(I\cap J)\subseteq \Rad(IJ).
    \]Concluimos con la iguadad\[
    \Rad(IJ)=\Rad(I\cap J).
    \]En particular,\[
    \Rad(I^n)=\Rad(I\cap\cdots\cap I)=\Rad(I).
    \]$\qed$
\end{enumerate}

\begin{definition}
    Sea $M\subseteq R$ un i-ideal (d-ideal). $M$ es \textit{i-maximal} (\textit{d-maximal}) si se satisfacen las dos condiciones siguientes\begin{enumerate}
        \item $M\neq R$.
        \item Si $M\subseteq L\subseteq R$, siendo $L$ un i-ideal (d-ideal), entonces $L=R$ o $L=M$.
    \end{enumerate}
Al conjunto de i-ideales i-maximales (d-ideales d-maximales) de $R$ se lo denota por $\Max_i(R)$ ($\Max_d(R)$). Al conjunto de ideales maximales de $R$ se lo denota por $\Max(R)$.
\end{definition}

Nótese que un ideal $M$ de $R$ es maximal si y sólo si los únicos ideales de $R/M$ son el 0 y $R/M$. En particular, si $R$ es conmutativo, un ideal es maximal si y sólo si $R/M$ es cuerpo. Esto es falso si $R$ es conmutativo: basta considerar cualquier anillo de división no conmutativo $D$ y obtenemos que $R=M_n(D)$ no tiene ideales (sí tiene i-ideales y d-ideales). Esto se debe a que cualquier ideal no nulo de $R$ contiene a todos los $e_{ij}$, como el lector comprobará fácilmente. Y por tanto, el ideal se trata de el anillo entero, $R$. Así, el 0 es ideal maximal, pero $R/0$ es un anillo isomorfo a $R$, que no es un cuerpo.

Por otro lado, notamos que si $M$ es un ideal maximal de $R$ e $I$ es otro ideal que no está contenido en $M$, entonces $M+I$ es un ideal que contiene estrictamente a $R$, lo que, por la maximalidad de $M$, implica $R=M+I$.

\begin{prop}
    Todo ideal maximal es primo.
\end{prop}
\noindent\textit{Demostración.} Sea $M$ un ideal maximal de $R$, y supongamos que $A,B$ son ideales de $R$ tales que $AB\subseteq M$. Si ninguno de los dos ideales estuviera contenido en $M$, por el párrafo previo a esta proposición se tiene\[
R=M+A=M+B.
\]Por tanto,\[
R=R^2=(M+A)(M+B)\subseteq M+AB\subseteq M,
\]lo que contradice la maximalidad de $M$.

Veamos algunos ejemplos:\begin{enumerate}
    \item No todo ideal primo es maximal. Por ejemplo, el idea primo $0\in\Spec{\Z}$ no es maximal.
    \item Sea $K$ un cuerpo. Sea $R=K[x,y]$. Considérese el epimorfismo de anillos\[
    \begin{array}{rrcl}
        \varphi:&R&\longrightarrow&K\\
        & f(x,y)&\longmapsto& f(0,0).
    \end{array}
    \]Entonces $\ker\varphi=(x,y)$, lo que implica\[
    K[x,y]/(x,y)\cong K
    \]y dado que $K$ es cuerpo, el ideal $(x,y)$ es maximal.

    \item Considérese el anillo\[
    R\coloneq\{f:[0,1]\longrightarrow\R|\;f\text{ es continua.}\}\subseteq\R^{[0,1]}.
    \]Sea $\alpha\in[0,1]$. Considérese el homomorfismo de anillos\[
    \begin{array}{rrcl}
        \varphi_\alpha:&R&\longrightarrow&\R\\
        f&\longmapsto&f(\alpha),
    \end{array}
    \]cuyo núcleo es $M(\alpha)\coloneq \ker(\varphi_\alpha)=\{f\in R|\; f(\alpha)=0\}$. Este ideal es maximal porque $R/M(\alpha)\cong R$.
\end{enumerate}

\begin{theorem}
    Sea $I\subseteq R$ un ideal (i-ideal, d-ideal) tal que $I\neq R$, siendo $R$ un anillo con identidad. Entonces existe un ideal (i-ideal, d-ideal) $M\subseteq R$ maximal (i-maximal, d-maximal) que contiene a $I$.
\end{theorem}
\noindent\textit{Demostración.} Lo probaremos para i-ideales y usaremos el lema de Zorn. Considérese el siguiente conjunto, que está ordenadado por inclusión y que es no vacío por contener a $I$:\[
\tau\coloneq\{J\subseteq R\text{ i-ideal}|\; I\subseteq J\neq R\}
\]Sea $\{C_k\}_{k\in K}$ una cadena de $\tau$. Considérese el siguiente candidado a cota superior:\[
C\coloneq \bigcup_{k\in K} C_k.
\]Es conocido que $C$ es un i-ideal, y este contiene a $I$ claramente. Supongamos por reducción al absurdo que $C=R$. Entonces $1\in C$. Entonces $1\in C_k$ para algún $k\in K$. Esto implica $C_k=R$, lo que no es posible. Así, $C\neq R$. Concluimos $C\in\tau$. Además, es cota superior. Por el lema de Zorn, existe un elemento maximal $M$ de $\tau$. Esto prueba el resultado. $\qed$

\begin{definition}
    El \textit{radical de Jacobson} de $R$ se define de la siguiente forma:\[
    \J(R)\coloneq\bigcap{M\in\Max_i(R)}M.
    \]
\end{definition}
Veamos un ejemplo: Si $R$ es un anillo y $M$ es un ideal maximal, entonces\[
S=R/M^n
\]Supongamos que $P/M^n$ es un ideal primo de $S$. Entonces $P$ es primo en $R$ y $M^n\subseteq P$. Como $P$ es primo, necesariamente $M\subseteq P$. Por la maximalidad de $M$, o $P=R$ o $P=M$. Por la primalidad de $P$, $P\neq R$. Concluimos así que $P=M$. Obtenemos el siguiente resultado:\[
\Spec\left(R/M^n\right)=\left\{M/M^n\right\}.
\]En particular,\[
\J(R/M^n)=M/M^n\neq 0.
\]

\begin{theorem}
    Sea $R$ un anillo. Son equivalentes:\begin{enumerate}
        \item $y\in J(R)$.
        \item Para todo $x\in R$, $1-xy$ tiene un inverso en $R$.
        \item Para todo $x,z\in R$, $1-xyz$ tiene un inverso en $R$.
    \end{enumerate}
\end{theorem}
\noindent\textit{Demostración.} Veamos que 1 implica 2.

\end{document}